\section{A NUMA-Aware GPU Runtime}

Current GPU software and hardware is co-designed together to optimize 
throughput of processors based on the assumption of uniform memory properties 
within the GPU. Fine grained interleaving of memory addresses across memory 
channels on the GPU provides implicit load balancing across memory but destroys 
memory locality.  As a result, thread block (CTA) scheduling policies need not 
be sophisticated to capture locality, which has been destroyed by the 
memory system layout.  For future NUMA GPUs to work well, both software and 
hardware must be changed to achieve both functionality and performance.  Before 
focusing on architectural changes to build a NUMA-aware GPU we describe the GPU 
runtime system we employ to enable multi-socket GPU execution.

Prior work has demonstrated it is possible to design a framework and a runtime 
system that transparently decomposes GPU kernels in sub-kernels and executes 
them on multiple PCIe attached GPUs in parallel~\cite{Cabezas2015}. For example, on NVIDIA GPUs 
this can be implemented by intercepting and remapping each kernel call, GPU 
memory allocation, memory copy, and GPU-wide synchronization issued by the CUDA 
driver. Special care needs to ensure that per-GPU memory fences are promoted to 
system level and seen by all GPUs as well as guaranteeing that sub-kernels CTA 
identifiers are properly managed to reflect those of the original kernel. In 
~\cite{Cabezas2015} these two problems were solved by introducing  code 
annotations and an additional source-to-source compiler which was also 
responsible for statically partitioning data placement and computation.

In our work, we follow a similar strategy but without using an source-to-source
translation. Unlike prior work, we are able to rely on NVIDIA's Unified Virtual 
Addressing~\cite{UVM} to allow dynamic placement of pages into memory at
runtime rather than static memory placement.  Similarly, technologies with
cache line granularity interconnects like NVIDIA's NVLink~\cite{NVLINK} allow
transparent access to remote memory without the need to modify application
source code to access local or remote memory addresses.  Due to these
advancements, we assume that through dynamic compilation of PTX to SASS at
executions the GPU runtime will be able to statically identify and promote 
system wide memory fences as well as manage sub-kernel CTA identifiers.

Current GPUs perform fine grained memory interleaving at a sub-page granularity 
across memory channels.  In a NUMA GPU this policy would destroy locality and 
result in 75\% of all accesses to be to remote memory in a 4 GPU system, an 
undesirable effect in NUMA systems.  Similarly, a round-robin page level 
interleaving could be utilized, similar to the Linux INTERLEAVE page allocation 
strategy, but despite the inherent memory load balancing, this still results in 
75\% of memory accesses occurring over low bandwidth NUMA links. Instead we 
leverage UVM page migration functionality to migrate pages on-demand from system 
memory to local GPU memory as soon as the first access (also called first-touch 
allocation) is performed as described by Arunkumar et. al~\cite{Arunkumar2017}.

On a single GPU fine grain dynamic assignment of CTAs to SMs is performed to 
achieve good load balancing.  Extending this policy to a multi-socket GPU 
system is not possible due to the relatively high latency of passing sub-kernel 
launches from software to hardware.  To overcome this penalty the GPU runtime 
must launch a block of CTAs to each GPU-socket at course granularity.  To 
encourage load balancing, each sub-kernel could be comprised of an interleaving 
of CTAs using modulo arithmetic.  Alternatively a single kernel can be 
decomposed into N sub-kernels, where N is the total number of GPU sockets in 
the system, assigning equal amount of contiguous CTAs to each GPU.  This design 
choice potentially exposes workload unbalance across sub-kernels, but it has 
been shown to preserve data locality present in applications where contiguous 
CTAs also access contiguous memory regions~\cite{Cabezas2015,Arunkumar2017}.

\subsection{Performance Through Locality} 

Figure~\ref{fig:motivation} shows the relative performance of a 4-socket NUMA 
GPU with respect to a single GPU under the two possible CTA scheduling and 
memory placement strategies explained above.  The green bars show the 
relative performance of traditional single GPU scheduling and memory 
interleaving policies when adapted to a NUMA GPU. The light blue bars
show the relative performance of using locality optimized GPU scheduling and 
memory placement, consisting of contiguous block CTA scheduling and first touch 
page migration. We can clearly see that the \textit{Locality-Optimized} 
solution almost always outperforms the traditional GPU scheduling and memory 
interleaving.  Without these runtime locality optimizations, in average a 4-socket NUMA 
GPU is not able to even match the performance of a single GPU despite the large 
increase in hardware resources.  Thus, using variants of prior 
proposals~\cite{Cabezas2015,Arunkumar2017}, we now only consider this locality 
optimized GPU runtime for the remainder of the paper.

Despite the performance improvements that can come via locality optimized 
software runtimes, many applications are not scaling well on our proposed NUMA 
GPU system. To illustrate this, Figure~\ref{fig:motivation} shows the workload 
throughput achievable by a hypothetical (unbuildable) 4$\times$ larger GPU with 
a red dash. This red dash represent an approximation of the 
maximum theoretical performance we could expect from a perfectly architected 
(both HW and SW) NUMA GPU system. Figure~\ref{fig:motivation} sorts the 
applications by the gap between relative performance of the Locality-Optimized 
NUMA GPU and hypothetical 4$\times$ larger GPU. We observe that on the right 
side of the graph some workloads (shown in the grey box) can achieve or surpass 
the maximum theoretical performance. In particular for the two far-most 
benchmarks on the right, the locality optimized solutions can outperform the 
hypothetical 4x larger GPU due higher cache hitrates because contiguous block 
scheduling is more cache friendly than traditional GPU scheduling.

However, for the applications on the left side there is a large gap between the 
Locality-Optimized NUMA design and theoretical performance. These are workloads 
in which either locality does not exist or the Locality-Optimized GPU runtime is 
not effective, resulting in large amount of remote data accesses still 
occurring.  Because our goal is to provide scalable performance for single GPU 
optimized applications, in the rest of the paper we aim to close this 
performance gap through microarchitectural innovation. To simplify later 
discussion, we choose to exclude benchmarks that achieve $\geq$99\% of the 
theoretical performance with SW-only locality optimizations. However, we include 
all benchmarks in our final results to show the overall performance scalability 
achievable with NUMA-aware multi-socket GPUs.
 
\subsection{Simulation Methodology}
\label{sec:methodology}

To evaluate the performance of future NUMA-aware multi-socket GPUs we use a 
proprietary, cycle-level, trace-driven simulator for single and multi-GPU 
systems. Our baseline GPU in both single GPU and multi-socket GPU 
configurations, approximates the latest NVIDIA Pascal 
architecture~\cite{pascal-tesla-wp}. Each streaming multiprocessors (SM) is 
modeled as an in-order processor with multiple levels of cache hierarchy 
containing private, per-SM, L1 caches and multi-banked, shared, L2 cache. Each 
GPU is backed by local on-package high bandwidth memory~\cite{HBM}. Our 
multi-socket GPU systems contains two to eight of these GPUs interconnected 
through a full bandwidth GPU switch as shown on Figure~\ref{fig:systemdiagram}. 
Table~\ref{tab:setup} provides a more detailed overview of the simulation 
parameters.

We study the scalability of multi-socket NUMA GPUs using 41 workloads 
taken from a range of 
production codes based on the HPC CORAL benchmarks~\cite{coral}, graph 
applications from Lonestar~\cite{lonestar}, HPC applications from 
Rodinia~\cite{Che2009}, in addition to several other in-house CUDA benchmarks. 
This set of workloads covers a wide spectrum of GPU applications used in 
machine learning, fluid dynamic, image manipulation, graph traversal, and
scientific computing.  Each of our benchmarks is hand selected to identify
the region of interest deemed representative for each workload, which may
be as small as a single kernel containing a tight inner loop or several thousand
kernel invocations.  We run each benchmark to completion for the determine
region of interest. Table~\ref{tab:numctas} provides both the memory footprint
of these workloads as well as the average number of active CTAs in the workload
(weighted by the time spent on each kernel) to provide a representation of 
how many parallel thread blocks (CTAs) are generally available
during workload execution.

\begin{table}[tp]
\begin{small}
\centering
\begin{tabular}{ll}
\toprule
\textbf{Parameter} & \textbf{Value(s)} \\
\toprule
Num of GPU sockets & 4 \\
\midrule
Total number of SMs & 64 per GPU socket \\
\midrule
GPU Frequency & 1GHz \\
\midrule
Max number of Warps & 64 per SM \\
\midrule
Warp Scheduler & Greedy then Round Robin \\
\midrule
L1 Cache & Private, 128 KB per SM, 128B lines, 4-way, \\ 
& GPU-side SW-based coherency \\
\midrule
L2 Cache & Shared, 4MB per socket, 128B lines, 16-way, \\ 
& Memory-side non-coherent\\
\midrule
GPU--GPU Interconnect & 128GB/s per socket (64GB/s each direction) \\
& 8 lanes 8B wide each per direction \\
&128-cycle latency \\
\midrule
DRAM Bandwidth & 768GB/s per GPU socket\\
\midrule
DRAM Latency & 100ns \\
\toprule
\end{tabular}
\vspace{-.1in}
\caption{Simulation parameters for evaluation of single and multi-socket GPU 
systems.}
\vspace{-.2in}
\label{tab:setup}
\end{small}
\end{table} 


\begin{table}[t]
\begin{small}
\centering
\begin{tabular}{lcc}
 \toprule
 \textbf{Benchmark} & \textbf{Time-weighted} & 
\textbf{Memory} \\
& \textbf{Average CTAs} & \textbf{Footprint (MB)} \\
 \toprule
ML-GoogLeNet-cudnn-Lev2 & 6272 & 1205 \\
ML-AlexNet-cudnn-Lev2 & 1250 & 832 \\
ML-OverFeat-cudann-Lev3 & 1800 & 388 \\
ML-AlexNet-cudnn-Lev4 & 1014 & 32 \\
ML-AlexNet-ConvNet2 & 6075 & 97 \\
Rodinia-Backprop & 4096 & 160 \\
Rodinia-Euler3D & 1008 & 25 \\
Rodinia-BFS  & 1954 & 38 \\
Rodinia-Gaussian  & 2599 & 78 \\
Rodinia-Hotspot  & 7396 & 64 \\
Rodinia-Kmeans  & 3249 & 221 \\
Rodnia-Pathfinder  & 4630 & 1570 \\
Rodinia-Srad  & 16384 & 98 \\
HPC-SNAP  & 200 & 744 \\
HPC-Nekbone-Large & 5583 & 294 \\
HPC-MiniAMR & 76033 & 2752 \\
HPC-MiniContact-Mesh1  & 250 & 21 \\
HPC-MiniContact-Mesh2  & 15423 & 257 \\
HPC-Lulesh-Unstruct-Mesh1  & 435 & 19 \\
HPC-Lulesh-Unstruct-Mesh2  & 4940 & 208 \\
HPC-AMG & 241549 & 3744 \\
HPC-RSBench  & 7813 & 19 \\
HPC-MCB  & 5001 & 162 \\
HPC-NAMD2.9  & 3888 & 88 \\
HPC-RabbitCT  & 131072 & 524 \\
HPC-Lulesh  & 12202 & 578 \\
HPC-CoMD  & 3588 & 319 \\
HPC-CoMD-Wa  & 13691 & 393 \\
HPC-CoMD-Ta  & 5724 & 394 \\
HPC-HPGMG-UVM  & 10436 & 1975 \\
HPC-HPGMG  & 10506 & 1571 \\
%HPC-HPGMG-UVM-Base & 359 & 10728 & 1975 \\
Lonestar-SP & 75 & 8 \\
Lonestar-MST-Graph & 770 & 86 \\
Lonestar-MST-Mesh & 895 & 75 \\
%HPC-Nekbone-medium & 510 & 3093 & 170 \\
Lonestar-SSSP-Wln & 60 & 21 \\
Lonestar-DMR  & 82 & 248 \\
Lonestar-SSSP-Wlc  & 163 & 21 \\
Lonestar-SSSP  & 1046 & 38 \\
Other-Stream-Triad  & 699051 & 3146 \\
Other-Optix-Raytracing  & 3072 & 87 \\
Other-Bitcoin-Crypto  & 60 & 5898 \\
\toprule
\end{tabular}
\vspace{-.1in}
\caption{Application footprint and average number of CTAs (thread blocks) 
available during time-weighted execution.}
\vspace{-.2in}
\label{tab:numctas}
\end{small}
\end{table}


% \begin{table}[t]
% \begin{small}
% \centering
% \begin{tabular}{lccc}
%  \toprule
%  \textbf{Benchmark} & \textbf{Kernels} & \textbf{Time-weighted} & 
% \textbf{Memory} \\
% & & \textbf{Average CTAs} & \textbf{(MB)} \\
%  \toprule
% ML-GoogLeNet-cudnn-Lev2 & 1 & 6272 & 1205 \\
% ML-AlexNet-cudnn-Lev2 & 1 & 1250 & 832 \\
% ML-OverFeat-cudann-Lev3 & 1 & 1800 & 388 \\
% ML-AlexNet-cudnn-Lev4 & 1 & 1014 & 32 \\
% ML-AlexNet-ConvNet2 & 1 & 6075 & 97 \\
% Rodinia-Backprop & 2 & 4096 & 160 \\
% Rodinia-Euler3D & 346 & 1008 & 25 \\
% Rodinia-BFS & 24 & 1954 & 38 \\
% Rodinia-Gaussian & 510 & 2599 & 78 \\
% Rodinia-Hotspot & 1 & 7396 & 64 \\
% Rodinia-Kmeans & 3 & 3249 & 221 \\
% Rodnia-Pathfinder & 20 & 4630 & 1570 \\
% Rodinia-Srad & 4 & 16384 & 98 \\
% HPC-SNAP & 118 & 200 & 744 \\
% HPC-Nekbone-Large & 300 & 5583 & 294 \\
% HPC-MiniAMR & 33 & 76033 & 2752 \\
% HPC-MiniContact-Mesh1 & 500 & 250 & 21 \\
% HPC-MiniContact-Mesh2 & 127 & 15423 & 257 \\
% HPC-Lulesh-Unstruct-Mesh1 & 2000 & 435 & 19 \\
% HPC-Lulesh-Unstruct-Mesh2 & 200 & 4940 & 208 \\
% HPC-AMG & 88 & 241549 & 3744 \\
% HPC-RSBench & 1 & 7813 & 19 \\
% HPC-MCB & 1 & 5001 & 162 \\
% HPC-NAMD2.9 & 1 & 3888 & 88 \\
% HPC-RabbitCT & 1 & 131072 & 524 \\
% HPC-Lulesh & 105 & 12202 & 578 \\
% HPC-CoMD & 350 & 3588 & 319 \\
% HPC-CoMD-Wa & 350 & 13691 & 393 \\
% HPC-CoMD-Ta & 350 & 5724 & 394 \\
% HPC-HPGMG-UVM & 359 & 10436 & 1975 \\
% HPC-HPGMG & 317 & 10506 & 1571 \\
% HPC-HPGMG-UVM-Base & 359 & 10728 & 1975 \\
% Lonestar-SP & 11 & 75 & 8 \\
% Lonestar-MST-Graph & 87 & 770 & 86 \\
% Lonestar-MST-Mesh & 71 & 895 & 75 \\
% HPC-Nekbone-medium & 510 & 3093 & 170 \\
% Lonestar-SSSP-Wln & 1000 & 60 & 21 \\
% Lonestar-DMR & 3 & 82 & 248 \\
% Lonestar-SSSP-Wlc & 1300 & 163 & 21 \\
% Lonestar-SSSP & 102 & 1046 & 38 \\
% Other-Stream-Triad & 5 & 699051 & 3146 \\
% Other-Optix-Raytracing & 1 & 3072 & 87 \\
% Other-Bitcoin-Crypto & 1 & 60 & 5898 \\
% \toprule
% \end{tabular}
% \caption{Application footprint and average number of CTAs (thread blocks) 
% available during time-weighted execution.}
% \label{tab:numctas}
% \end{small}
% \end{table}
