\section{Related Work}
Multi-GPUs are widely used for scaling single GPU perfromance
via integration of multiple GPUs at the system
level~\cite{pascal-tesla-wp,dgx,intersect360,titan_supercomputer} for a
rapidly growing pool of applications~\cite{coral,cudnn,Lavin15b,SimonyanZ14a}.
Similarly, multi-socket and multi-node CPU
installations have been employed and studied in context of HPC and
data-center applications~\cite{Intel:Xeon,IBM:Power,IBM:z196,AMD:Opteron}

Multi-GPU programming models require explicit programming of multiple GPUs
using SW APIs such as Peer-2-Peer access~\cite{NVIDIAP2P} or a combination of
MPI and CUDA~\cite{NVIDIAMPI} to manage multiple GPUs. These extensions require
unique programming experience and non-negligible SW effort while adapting a
single GPU application to take advantage of a Multi-GPU system. In this paper
we execute single GPU application on a Multi-GPU system as if it was a single
larger GPU via hardware innovations and extensions to the driver software to
provide programmer- and OS-transparent execution, similarly to approaches
proposed in the past~\cite{Cabezas2015,lee2013transparent,ben2015memory}.

Modern multi-socket CPU and GPU systems leverage most advanced interconnect
technologies such as Nvidia NVLink, Intel QPI and AMD
Infinity~\cite{dgx,INTELQPI,AMDINFINITYFABRIC}. These modern fabrics utilize
high speed serial signalling technologies over unidirectional lanes
collectively comprizing full-duplex links. This way each link capacity is
statically allocated at design time and usually is symmetric in nature. In this
paper we propose to dynamically re-allocate available link bandwidth resources
by using same system wire resources and on-chip I/O interfaces, while
implementing both recever and transmitter driver circuitry at each lane. This
approach resembles previously proposed tri-state bi-directional bus
technologies ~\cite{tri-state}, or former technologies such as Intel front-side
bus ~\cite{fsb}. Our proposal however allows leveraging fast
singled ended signalling, while allowing a dynamically controlled
asymmetric bandwidth allocation via on-the-fly reconfiguration of the
individual lane direction within a link.

Static and dynamic cache partitioning techniques were widely explored in
context of CPU caches and QoS
~\cite{ics2007,Herdrich2016CacheQF,pact06,qureshi-micro,jaleel-pact} For
example, Rafique et al~\cite{pact06} proposed architectural support for shared
cache management with quota-based approach. Qureshi et al~\cite{qureshi-micro}
proposed to partition the cache space between applications. Jaleel et
al~\cite{jaleel-pact} improved on this by proposing adaptive insertion
policies. Recently, cache monitoring and allocation technologies were added to
Intel Xeon procesors, targeted for QoS enforcement via dynamic repartitioning
of on-chip CPU cache resources~\cite{Herdrich2016CacheQF}.  Efficient cache
partitioning in GPU was primarly explored in context of L1 caches
~\cite{li-priority-based}. While dynamic cache paritioning has been widely used for
QoS and L1 utlization, to the best of our knowledge it has never been used
to try to optimize performance when caches are backed with NUMA memory
systems.